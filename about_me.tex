\documentclass{resume}
\name{Eric Kim}
\addressone{1684 Oxford Street}
\addresstwo{Berkeley, CA 94709}
\email{eric.kim.cs@gmail.com}
\phone{805-300-9474}
\website{http://www.eric-kim.net}

\usepackage{verbatim}

\begin{document}

\maketitle
\thispagestyle{empty} %% no page numbers

\begin{component}{About Me}
I am a Berkeley graduate interested in pursuing a PhD degree in Computer Science. Despite having discovered Computer Science relatively late in the college timeline, I quickly developed a keen interest in the field, and earnestly devoted myself to the studies. Wishing to examine the material at a deeper level, I successfully pursued several undergraduate teaching opportunities, and in addition to gaining valuable teaching experience, developed a strong interest to continue teaching. 

In addition, my desire to seek experience beyond the classroom led me to work with Professor David Wagner's ``Electronic Voting and Computer Vision'' research group, where we developed several tools, performed pilot audits with various California counties, and published at EVT/WOTE. My work with the research group sparked and fostered an interest in Computer Vision. 

Due to my joint desire to continue working with Computer Vision and teaching, I believe that pursuing a PhD degree is the best fit for me.
\end{component}

\vspace{-0.5em}

\begin{component}{Education}
	\school{University of California, Berkeley}{Computer Science}{B.A.}{December 2011}
\end{component}

\begin{component}{Research Experience}
    \begin{position}{Research Engineer}{January 2012 -- Present}
        {University of California, Berkeley}{Department of Computer Science}
    {Designed and developed election auditing software, whose scope includes the application of Computer Vision and Image Processing techniques. Successfully performed several pilot audit programs in various California counties. Will soon be processing November 2012 election data.}
    \end{position}
    
    \begin{position}{Research Assistant}{August 2010 -- January 2012}
        {University of California, Berkeley}{Department of Computer Science}
    {}
    \end{position}
\end{component}

\begin{component}{Teaching Experience}
    \textbf{Teaching Assistant (CS 61A)} \hfill May 2012 -- August 2012 \\
    \textbf{Teaching Assistant (CS 61A, John DeNero)} \hfill August 2011 -- December 2011 \\
    \textbf{Teaching Assistant (CS 3L)} \hfill May 2011 -- August 2011 \\
    \textbf{Teaching Assistant (CS 61A, Brian Harvey)} \hfill January 2011 -- May 2011 \\
    \textbf{Teaching Assistant (CS 61A, Brian Harvey)} \hfill August 2010 -- December 2010 \\
    \textbf{Teaching Assistant (CS 61BL)} \hfill May 2010 -- August 2010 \\
        \textit{University of California, Berkeley \hfill Department of Computer Science}\\
    Taught several key undergraduate Computer Science courses. Duties included holding weekly sections, writing and grading exams, holding office hours, and developing course materials.
\end{component}

\begin{component}{Academic Papers}
\vspace{0.5em}
``Operator-Assisted Tabulation of Optical Scan Ballots,'' Kai Wang, Eric Kim, Nicholas Carlini, Ivan Motyashov, Daniel Nguyen, David Wagner. \emph{EVT/WOTE 2012}, August 2012.
        \begin{itemize}
        \vspace{-0.5em}\item[] Developed a tool that allows an operator
to accurately and efficiently tabulate scanned ballots from an election,
via a system that interleaves Computer Vision algorithms and focused
operator assistance.
        \end{itemize}

``An Analysis of Write-in Marks on Optical Scan Ballots,'' Theron Ji, Eric Kim, Raji Srikantan, Alan Tsai, Arel Cordero, and David Wagner. \emph{EVT/WOTE 2011}, August 2011.
	\begin{itemize}
	\vspace{-0.5em}\item[] Developed a system to achieve automatic recognition of write-in marks on marked voter ballots. Evaluated the system on
				       a large-scale election in Leon County, Florida, and studied the kinds of write-in marks that are seen in practice.
	\end{itemize}
\end{component}

\begin{component}{Relevant Coursework}
	\begin{tabularfw}{l c r}
	Algorithms & Discrete Math & Data Structures \\
	Machine Structures & Operating Systems & Artificial Intelligence \\
	Machine Learning & Compilers & Computer Security
	\end{tabularfw}
\end{component}

\begin{comment}
\begin{component}{Relevant Coursework}
	\begin{tabularfw}{l c c r}
	Algorithms & Discrete Math & Data Structures & Machine Structures \\
	Operating Systems & Artificial Intelligence & Machine Learning \\
	Compilers & Computer Security
	\end{tabularfw}
\end{component}
\end{comment}

\end{document}

