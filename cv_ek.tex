\documentclass{resume}
\name{Eric Kim}
\addressone{417 1/2 Veteran Avenue}
\addresstwo{Los Angeles, CA 90024}
\email{eric.kim.cs@gmail.com}
\phone{805-300-9474}
\website{http://www.eric-kim.net}

\usepackage{verbatim}

\begin{document}

\maketitle
\thispagestyle{empty} %% no page numbers

\vspace{-0.5em}

\begin{component}{Research Interests}
Computer vision, machine learning, medical imaging.
%I am interested in developing principled methods to
%analyze image and video streams.
%I am interested in image representation and model-based Vision.
%In particular, I would like
%to develop methods that can scale to the billions of images available on the Internet.
%This includes unsupervised methods that can take effectively advantage of the large corpus of
%images without the expensive requirement of high-quality labels.
Specific applications: face recognition, tissue segmentation, visual modeling, and 3D reconstruction.
\end{component}

\vspace{0.5em}

\begin{component}{Education}
    \school{University of California, Los Angeles}{Computer Science}{M.S.}{09/2013 -- 06/2016}

    \hspace{1em} Advisors: Professor Demetri Terzopoulos, Dr. M. Alex O. Vasilescu (Tensor Vision Technologies)

    \hspace{1em} Thesis: ``A Part-Based, Multiresolution, TensorFaces Approach to Image-Based Facial Verification''

	\school{University of California, Berkeley}{Computer Science}{B.A.}{08/2007 -- 12/2011}
\end{component}

\vspace{0.5em}

\begin{component}{Publications}
\vspace{0.5em}
``Improved Support for Machine-Assisted Ballot-Level Audits,'' Eric Kim, Nicholas Carlini, Andrew Chang, George Yiu, Kai Wang, David Wagner. \emph{EVT/WOTE 2013}, August 2013.
    \begin{itemize}
        \vspace{-0.5em}\item[] Presented improvements to the OpenCount system,
        which significantly reduce operator effort.
        Our new techniques yield order-of-magnitude speedups compared
        to the previous system, and enable us to successfully process
        elections that would not have reasonably feasible without these
        improvements.
    \end{itemize}
    
``Operator-Assisted Tabulation of Optical Scan Ballots,'' Kai Wang, Eric Kim, Nicholas Carlini, Ivan Motyashov, Daniel Nguyen, David Wagner. \emph{EVT/WOTE 2012}, August 2012.
        \begin{itemize}
        \vspace{-0.5em}\item[] Designed and developed the OpenCount software, which allows an operator
to efficiently create vote tallies of scanned voted ballots from an election. This is accomplished via a system 
that interleaves computer vision algorithms and operator assistance to achieve perfect (or near-perfect) accuracy.
Successfully applied the tool on several California pilot audits in 2011. 
        \end{itemize}

``An Analysis of Write-in Marks on Optical Scan Ballots,'' Theron Ji, Eric Kim, Raji Srikantan, Alan Tsai, Arel Cordero, and David Wagner. \emph{EVT/WOTE 2011}, August 2011.
	\begin{itemize}
	\vspace{-0.5em}\item[] Developed a system to achieve automatic recognition of write-in marks on marked voter ballots. Evaluated the system on
				       a large-scale election in Leon County, Florida, and studied the kinds of write-in marks that are seen in practice.
	\end{itemize}
\end{component}

\vspace{-0.5em}

\begin{component}{Posters}

``OpenCount: Operator-Assisted Tabulation of Optical Scan Ballots'', Eric Kim, Nicholas Carlini, Andrew Chang, George Yiu, Zongheng Yang, Kai Wang, David Wagner. \emph{NIST Future of Voting Systems Symposium}, February 2013. 
\end{component}

\vspace{0.5em}

\begin{component}{Research Experience}
    \begin{position}{Graduate Researcher}{September 2014 -- June 2016}
        {University of California, Los Angeles}{Department of Computer Science}
    \emph{Tensor Vision Technologies}

    {Developed a novel facial verification system. 
By analyzing faces in a multiresolution, part-based multilinear framework, we improved verification results by 13\% on the ``Labeled Faces in the Wild'' dataset relative to a previous multilinear approach (79\% overall).

This work matured into my MS thesis, titled: ``A Part-Based, Multiresolution, TensorFaces Approach to Image-Based Facial Verification''.
Advisors: Professor Demetri Terzopoulos (UCLA), Dr. M. Alex O. Vasilescu (Tensor Vision Technologies).
}
    \end{position}

    \begin{position}{Research Programmer}{January 2016 -- Present}{University of California, Los Angeles}{School of Dentistry}
      {
Developed a statistical model of shape and appearance to perform bone contour segmentation of 3D imaging data.
The algorithm iteratively improves the contour by updating each landmark based on a learned model of local appearance, followed by a global shape constraint update.
\\
Applied 3D mesh algorithms to quantitatively determine facial surgery effects on facial structure.
After achieving mesh correspondence by applying a nonrigid iterative closest point registration algorithm, I ran statistical tests on pre/post operation facial structure data to determine statistically significant regions of change.
}
      \end{position}


    \begin{position}{Research Engineer}{January 2012 -- August 2013}
        {}{}{}
    \end{position}

\vspace{-3.25em}

    \begin{position}{Research Assistant}{August 2010 -- January 2012}
        {University of California, Berkeley}{Department of Computer Science}
    {Designed and developed the election auditing software ``OpenCount''.
Involved the use of computer vision and image processing techniques: image registration, digit recognition, automatic visual barcode decoding, and ``human-in-the-loop'' processing for efficient data entry.
Successfully performed several pilot audit programs in several California counties.}
    \end{position}
    
\end{component}

\vspace{-1.5em}

\begin{component}{Teaching Experience}
    \textbf{Teaching Assistant (CS 61A)} \hfill May 2012 -- August 2012 \\
    \textbf{Teaching Assistant (CS 61A)} \hfill August 2011 -- December 2011 \\
    \textbf{Teaching Assistant (CS 3L)} \hfill May 2011 -- August 2011 \\
    \textbf{Teaching Assistant (CS 61A)} \hfill January 2011 -- May 2011 \\
    \textbf{Teaching Assistant (CS 61A)} \hfill August 2010 -- December 2010 \\
    \textbf{Teaching Assistant (CS 61BL)} \hfill May 2010 -- August 2010 \\
        \textit{University of California, Berkeley \hfill Department of Computer Science}\\
    \textbf{Teaching Assistant (CS 33)} \hfill April 2015 -- June 2015 \\
    \textbf{Teaching Assistant (PIC 10A)} \hfill January 2016 -- March 2016 \\
    \textbf{Teaching Assistant (PIC 10A)} \hfill March 2016 -- June 2016 \\
        \textit{University of California, Los Angeles \hfill Department of Computer Science}\\
    Taught several undergraduate computer science courses. Duties included holding weekly sections, writing and grading exams, holding office hours, and developing course material.
\end{component}

\vspace{0.5em}

\begin{component}{Projects}
	\begin{itemize}
		\vspace{-0.5em}\item FourVoices: An automatic music generator. Utilizing artificial intelligence and music theory led to an elegant and extendable framework that consistently outputs pleasing music.
I transformed the problem into a set of constraints and variables, and solved the resulting problem with a general-purpose constraint satisfaction solver.
The project is hosted on GitHub, where I have also written documentation and tutorials.
		(Python)
        \vspace{-0.5em}\item Handwriting recognition. Utilized adaptive splines to recognize handwritten characters.
To recognize a handwritten character, a deformable spline model is fit to the character via an iterative deformation algorithm.
The algorithm outputs a deformation cost which is used for recognition: the label of the spline model with smallest cost is declared the output label.
(Matlab)
        \vspace{-0.5em}\item Efficient barcode decoder
	        for the Interleaved 2-of-5 format. (Python, OpenCV)
        %\vspace{-0.5em}\item Maintains a personal website. Topics include: 
        %computer science, machine learning, computer vision.
        \vspace{-0.5em}\item Wrote a gentle tutorial to kernel methods as used in machine learning. Title: ``\emph{Everything You Wanted to Know about the Kernel Trick (But Were Too Afraid to Ask)}''. This article is the second search result returned for Google searches of ``kernel trick'', as of 2016.
		\vspace{-0.5em}\item Python 2.5 compiler targeting the x86 ISA, with the addition of
		strong typing support. (C++, Python)
	\end{itemize}
\end{component}

\vspace{-1.0em}

\begin{component}{Key Skills}
	\textbf{Technical Skills}\\
		Programming Languages: Python, Matlab, C, C++, Java, Javascript, HTML, CSS, php, Scheme, Assembly (MIPS, x86\_64)\\
        Specializations: Face recognition, computer vision, medical imaging, machine learning, nonlinear optimization, automatic landmark detection, data analysis.\\
%		Environments: UNIX, Windows\\
		Libraries: OpenCV, vlfeat, numpy, scipy, OpenGL \\
		Productivity: Version control (svn, git, mercurial), GDB, Wireshark, UNIX toolset, LaTeX
\end{component}

\begin{component}{Graduate Coursework}
\begin{tabularfw}{l c r}
Machine Perception & Linear Programming & Convex Optimization \\
Reasoning with Partial Beliefs & Large Scale Optimization & Applied Probability \\
&Visual Modeling \\
\end{tabularfw}
\end{component}

\begin{component}{Undergraduate Coursework}
	\begin{tabularfw}{l c r}
	Algorithms & Discrete Math & Data Structures \\
	Machine Structures & Operating Systems & Artificial Intelligence \\
	Machine Learning & Compilers & Computer Security \\
    Linear Algebra & Computer Networks & Computer Graphics \\
	\end{tabularfw}
\end{component}

\begin{comment}
\begin{component}{Relevant Coursework}
	\begin{tabularfw}{l c c r}
	Algorithms & Discrete Math & Data Structures & Machine Structures \\
	Operating Systems & Artificial Intelligence & Machine Learning \\
	Compilers & Computer Security
	\end{tabularfw}
\end{component}
\end{comment}

\end{document}

