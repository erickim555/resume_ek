\documentclass{resume}
\name{Eric Kim}
\addressone{2246 8th Street, A}
\addresstwo{Berkeley, CA 94710}
\email{eric.kim.cs@gmail.com}
\phone{805-300-9474}
\website{http://www.eric-kim.net}

\usepackage{verbatim}

\begin{document}

\maketitle
\thispagestyle{empty} %% no page numbers

\begin{component}{Research Interests}
Computer Vision, applications of Machine Learning, and Electronic Voting.
In particular, I am interested in work that seeks to bridge the ``semantic
gap'' via algorithms that extract high-level information from an image
or video stream. 
\end{component}

\vspace{-0.5em}

\begin{component}{Education}
	\school{University of California, Berkeley}{Computer Science}{B.A.}{December 2011}
\end{component}

\begin{component}{Publications}
\vspace{0.5em}
``Operator-Assisted Tabulation of Optical Scan Ballots,'' Kai Wang, Eric Kim, Nicholas Carlini, Ivan Motyashov, Daniel Nguyen, David Wagner. \emph{EVT/WOTE 2012}, August 2012.
        \begin{itemize}
        \vspace{-0.5em}\item[] Developed a tool that allows an operator
to accurately and efficiently tabulate scanned ballots from an election,
via a system that interleaves Computer Vision algorithms and focused
operator assistance.
        \end{itemize}

``An Analysis of Write-in Marks on Optical Scan Ballots,'' Theron Ji, Eric Kim, Raji Srikantan, Alan Tsai, Arel Cordero, and David Wagner. \emph{EVT/WOTE 2011}, August 2011.
	\begin{itemize}
	\vspace{-0.5em}\item[] Developed a system to achieve automatic recognition of write-in marks on marked voter ballots. Evaluated the system on
				       a large-scale election in Leon County, Florida, and studied the kinds of write-in marks that are seen in practice.
	\end{itemize}
\end{component}

\begin{component}{Research Experience}
    \begin{position}{Research Engineer}{January 2012 -- Present}
        {University of California, Berkeley}{Department of Computer Science}
    {Designed and developed election auditing software, whose scope includes the application of Computer Vision and Image Processing techniques. Successfully performed several pilot audit programs in various California counties. Will soon be processing November 2012 election data.}
    \end{position}
    
    \begin{position}{Research Assistant}{August 2010 -- January 2012}
        {University of California, Berkeley}{Department of Computer Science}
    {}
    \end{position}
\end{component}

\vspace{-2.0em}

\begin{component}{Teaching Experience}
    \textbf{Teaching Assistant (CS 61A)} \hfill May 2012 -- August 2012 \\
    \textbf{Teaching Assistant (CS 61A)} \hfill August 2011 -- December 2011 \\
    \textbf{Teaching Assistant (CS 3L)} \hfill May 2011 -- August 2011 \\
    \textbf{Teaching Assistant (CS 61A)} \hfill January 2011 -- May 2011 \\
    \textbf{Teaching Assistant (CS 61A)} \hfill August 2010 -- December 2010 \\
    \textbf{Teaching Assistant (CS 61BL)} \hfill May 2010 -- August 2010 \\
        \textit{University of California, Berkeley \hfill Department of Computer Science}\\
    Taught several key undergraduate Computer Science courses. Duties included holding weekly sections, writing and grading exams, holding office hours, and developing course materials.
\end{component}

\begin{component}{Misc.}
	\begin{position}{Significant Projects}{Ongoing}
		{}{\vspace{-1.0em}}
	{Projects with significant scope and technical expertise for the purpose of applying concepts, in addition to
	sharpening programming skills.}

	\begin{itemize}
        \vspace{-0.5em}\item Implemented an efficient barcode decoder
        for the Interleaved 2-of-5 format. (Python, OpenCV)
		\vspace{-0.5em}\item Built a static analysis tool that, given the source code of a program that
		processes attacker-controlled input, finds an input that maximizes execution time, effectively
		creating a Denial-of-Service (DoS) attack vector. (C, Gametime tool)
		\vspace{-0.5em}\item Built a Python 2.5 compiler targeting the x86 ISA, with the addition of
		optional strong typing support. (C++, Bison variant, Python)
		\vspace{-0.5em}\item A symbolic music generator that provides harmonically-correct
		parts for SATB voices. Framing the problem as a CSP and as a Dynamic Programming problem led to 
        efficient algorithms.
		(Python)
		\vspace{-0.5em}\item Implemented various key low-level aspects of the 
		educational NachOS operating system, such as demand paging, multi-threading, virtual memory, and 
		networking.
		\vspace{-0.5em}\item Built a spam classifier, utilizing the Naive Bayes probabilistic model in order to make
		learning and inference stages tractable. (Python, numpy)
		\vspace{-0.5em}\item Implemented several AI algorithms for the game of Pacman, utilizing
		techniques such as: graph search, adversarial search, and reinforcement learning. (Python)
	\end{itemize}
	\end{position}
\end{component}

\vspace{-2.0em}

\begin{component}{Key Skills}
	\textbf{Technical Skills}\\
		Programming Languages: Python, Matlab, C, C++, Java, Javascript, HTML, CSS, php, Scheme, Assembly (x86)\\
		Environments: UNIX variants, Windows\\
		Productivity: Version control (svn, git, mercurial), GDB, Wireshark, UNIX toolset, LaTeX
\end{component}

\begin{component}{Relevant Coursework}
	\begin{tabularfw}{l c r}
	Algorithms & Discrete Math & Data Structures \\
	Machine Structures & Operating Systems & Artificial Intelligence \\
	Machine Learning & Compilers & Computer Security \\
    Linear Algebra
	\end{tabularfw}
\end{component}

\begin{comment}
\begin{component}{Relevant Coursework}
	\begin{tabularfw}{l c c r}
	Algorithms & Discrete Math & Data Structures & Machine Structures \\
	Operating Systems & Artificial Intelligence & Machine Learning \\
	Compilers & Computer Security
	\end{tabularfw}
\end{component}
\end{comment}

\end{document}

