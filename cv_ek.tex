\documentclass{resume}
\name{Eric Kim}
\addressone{1684 Oxford Street}
\addresstwo{Berkeley, CA 94709}
\email{eric.kim.cs@gmail.com}
\phone{805-300-9474}
\website{http://www.eric-kim.net}

\usepackage{verbatim}

\begin{document}

\maketitle
\thispagestyle{empty} %% no page numbers

\begin{component}{Objective}
To engineer and develop cutting-edge products with usability, efficiency, and scalability in mind, while also leveraging
the recent massive influx of data toward building better products. \\
\end{component}

\vspace{-0.5em}

\begin{component}{Education}
	\school{University of California, Berkeley}{Computer Science}{B.A.}{December 2011}
\end{component}

\begin{component}{Key Skills}
	\textbf{Technical Skills}\\
		Programming Languages: Python, C, C++, Java, Javascript, HTML, CSS, php, Scheme, Assembly (x86)\\
		Environments: UNIX variants, Windows\\
		Productivity: Version control (svn, git, mercurial), GDB, Wireshark, UNIX toolset
\end{component}

\begin{component}{Experience}
	\begin{position}{Research Assistant}{May 2011 -- Present}
	 	{University of California, Berkeley}{Department of Computer Science}
	{Developed election auditing software, whose scope includes Computer Vision and UI design, which 
	will be used to enable progressive election audit legislature in states across the US.}
	\end{position}

	\begin{position}{Teaching Assistant}{May 2010 -- December 2011}
		{University of California, Berkeley}{Department of Computer Science}
	{Taught several key undergraduate Computer Science courses: CS 3L, CS 61A, CS6BL. 
	 Duties included holding weekly sections, writing exams, grading, 
	holding office hours, and developing course materials.}
	\end{position}

	\begin{position}{Significant Projects}{Ongoing}
		{}{\vspace{-1.0em}}
	{Projects with significant scope and technical expertise for the purpose of applying concepts, in addition to
	sharpening programming skills.}

	\begin{itemize}
		\vspace{-0.5em}\item Built a static analysis tool that, given the source code of a program that
		processes attacker-controlled input, finds an input that maximizes execution time, effectively
		creating a Denial-of-Service (DoS) attack vector. (C, Gametime tool)
		\vspace{-0.5em}\item Built a Python 2.5 compiler targeting the x86 ISA, with the addition of
		optional strong typing support. (C++, Bison variant, Python)
		\vspace{-0.5em}\item A symbolic music generator that provides harmonically-correct
		parts for SATB voices. Framing the problem as a CSP and as a Dynamic Programming problem led to efficient generators.
		(Python)
		\vspace{-0.5em}\item Implemented various key low-level aspects of the 
		educational NachOS operating system, such as demand paging, multi-threading, virtual memory, and 
		networking.
		\vspace{-0.5em}\item Built a spam classifier, utilizing the Naive Bayes probabilistic model in order to make
		learning and inference stages tractable. (Python, numpy)
		\vspace{-0.5em}\item Implemented several AI algorithms for the game of Pacman, utilizing
		techniques such as: graph search, adversarial search, and reinforcement learning. (Python)
	\end{itemize}
	\end{position}
\end{component}
\vspace{-2.0em}
\begin{component}{Academic Papers}
\vspace{0.5em}
``An Analysis of Write-in Marks on Optical Scan Ballots,'' Theron Ji, Eric Kim, Raji Srikantan, Alan Tsai, Arel Cordero, and David Wagner. \emph{EVT/WOTE 2011}, August 2011.
	\begin{itemize}
	\vspace{-0.5em}\item[] Developed a system to achieve automatic recognition of write-in marks on marked voter ballots. Evaluated the system on
				       a large-scale election in Leon County, Florida, and studied the kinds of write-in marks that are seen in practice.
	\end{itemize}

``The Worst Video You Have Ever Seen: Exacerbating the Computation Time of Systems,'' Eric Kim, Jonathan Kotker, Sunil Pedapudi, Sanjit Seshia, and David Wagner. December 2011.
	\begin{itemize}
	\vspace{-0.5em}\item[] Demonstrated a novel attack in which an attacker analyzes the source code of a program in order to produce an input that 
				       maximizes execution time, creating a Denial-of-Service. Applied the GameTime tool, which statically analyzes
				       source code and determines execution times for program paths.
	\end{itemize}


\end{component}

\begin{component}{Relevant Coursework}
	\begin{tabularfw}{l c r}
	Algorithms & Discrete Math & Data Structures \\
	Machine Structures & Operating Systems & Artificial Intelligence \\
	Machine Learning & Compilers & Computer Security
	\end{tabularfw}
\end{component}

\begin{comment}
\begin{component}{Relevant Coursework}
	\begin{tabularfw}{l c c r}
	Algorithms & Discrete Math & Data Structures & Machine Structures \\
	Operating Systems & Artificial Intelligence & Machine Learning \\
	Compilers & Computer Security
	\end{tabularfw}
\end{component}
\end{comment}

\end{document}

