\documentclass{resume}
\name{Eric Kim}
\addressone{417 1/2 Veteran Avenue}
\addresstwo{Los Angeles, CA 90024}
\email{eric.kim.cs@gmail.com}
\phone{805-300-9474}
\website{http://www.eric-kim.net}

\usepackage{verbatim}

\begin{document}

\maketitle
\thispagestyle{empty} %% no page numbers

\vspace{-0.5em}

\begin{component}{Research Interests}
Computer vision, machine learning.
%I am interested in developing principled methods to
%analyze image and video streams.
I am interested in image representation and model-based Vision.
%In particular, I would like
%to develop methods that can scale to the billions of images available on the Internet.
%This includes unsupervised methods that can take effectively advantage of the large corpus of
%images without the expensive requirement of high-quality labels.
Specific applications include: face recognition, image understanding, and 3D reconstruction.
\end{component}

\vspace{0.5em}

\begin{component}{Education}
    \school{University of California, Los Angeles}{Computer Science}{M.S.}{09/2013 -- 06/2016}

    \hspace{2em} Advisors: Professor Demetri Terzopoulos, Dr. M. Alex O. Vasilescu (Tensor Vision Technologies)

	\school{University of California, Berkeley}{Computer Science}{B.A.}{08/2007 -- 12/2011}
\end{component}

\vspace{0.5em}

\begin{component}{Publications}
\vspace{0.5em}
``Improved Support for Machine-Assisted Ballot-Level Audits,'' Eric Kim, Nicholas Carlini, Andrew Chang, George Yiu, Kai Wang, David Wagner. \emph{EVT/WOTE 2013}, August 2013.
    \begin{itemize}
        \vspace{-0.5em}\item[] Presented improvements to the OpenCount system,
        which significantly reduce operator effort.
        Our new techniques yield order-of-magnitude speedups compared
        to the previous system, and enable us to successfully process some
        elections that would not have reasonably feasible without these
        techniques.
    \end{itemize}
    
``Operator-Assisted Tabulation of Optical Scan Ballots,'' Kai Wang, Eric Kim, Nicholas Carlini, Ivan Motyashov, Daniel Nguyen, David Wagner. \emph{EVT/WOTE 2012}, August 2012.
        \begin{itemize}
        \vspace{-0.5em}\item[] Designed and developed the OpenCount tool, which allows an operator
to efficiently create vote tallies of scanned voted ballots from an election. This is accomplished via a system 
that interleaves computer vision algorithms and operator assistance in order to achieve perfect (or near-perfect) accuracy.
Successfully applied the tool on several California pilot audits in 2011. 
        \end{itemize}

``An Analysis of Write-in Marks on Optical Scan Ballots,'' Theron Ji, Eric Kim, Raji Srikantan, Alan Tsai, Arel Cordero, and David Wagner. \emph{EVT/WOTE 2011}, August 2011.
	\begin{itemize}
	\vspace{-0.5em}\item[] Developed a system to achieve automatic recognition of write-in marks on marked voter ballots. Evaluated the system on
				       a large-scale election in Leon County, Florida, and studied the kinds of write-in marks that are seen in practice.
	\end{itemize}
\end{component}

\vspace{-0.5em}

\begin{component}{Posters}

``OpenCount: Operator-Assisted Tabulation of Optical Scan Ballots'', Eric Kim, Nicholas Carlini, Andrew Chang, George Yiu, Zongheng Yang, Kai Wang, David Wagner. \emph{NIST Future of Voting Systems Symposium}, February 2013. 
\end{component}

\vspace{0.5em}

\begin{component}{Research Experience}
    \begin{position}{Graduate Researcher}{September 2014 -- June 2016}
        {University of California, Los Angeles}{Department of Computer Science}
    \emph{Tensor Vision Technologies}

    {Developed a novel facial verification system within a multilinear framework. Utilized techniques from computer vision, machine learning, and multilinear analysis.}
    \end{position}

    \begin{position}{Programmer}{January 2016 -- Present}{University of California, Los Angeles}{School of Dentistry}
      Developed a system to perform automatic fiducial marker acquisition on 3D cone-beam computed tomography scans of human subjects for the purpose of 3D cephalometric analysis.
      \end{position}

    \begin{position}{Research Engineer}{January 2012 -- August 2013}
        {University of California, Berkeley}{Department of Computer Science}
    {Designed and developed election auditing software, whose scope includes the application of computer Vision and image processing techniques. Successfully performed several pilot audit programs in various California counties. Future work includes additional pilot audits, as well as improvements to the system itself.}
    \end{position}
    
    \begin{position}{Research Assistant}{August 2010 -- January 2012}
        {University of California, Berkeley}{Department of Computer Science}
    {}
    \end{position}
\end{component}

\vspace{-1.5em}

\begin{component}{Teaching Experience}
    \textbf{Teaching Assistant (CS 61A)} \hfill May 2012 -- August 2012 \\
    \textbf{Teaching Assistant (CS 61A)} \hfill August 2011 -- December 2011 \\
    \textbf{Teaching Assistant (CS 3L)} \hfill May 2011 -- August 2011 \\
    \textbf{Teaching Assistant (CS 61A)} \hfill January 2011 -- May 2011 \\
    \textbf{Teaching Assistant (CS 61A)} \hfill August 2010 -- December 2010 \\
    \textbf{Teaching Assistant (CS 61BL)} \hfill May 2010 -- August 2010 \\
        \textit{University of California, Berkeley \hfill Department of Computer Science}\\
    \textbf{Teaching Assistant (CS 33)} \hfill April 2015 -- June 2015 \\
    \textbf{Teaching Assistant (PIC 10A)} \hfill January 2016 -- March 2016 \\
    \textbf{Teaching Assistant (PIC 10A)} \hfill March 2016 -- June 2016 \\
        \textit{University of California, Los Angeles \hfill Department of Computer Science}\\
    Taught several undergraduate computer science courses. Duties included holding weekly sections, writing and grading exams, holding office hours, and developing course material.
\end{component}

\vspace{0.5em}

\begin{component}{Misc.}
	\begin{position}{Significant Projects}{Ongoing}
		{}{\vspace{-1.0em}}
	{Projects with significant scope and technical expertise for the purpose of applying concepts, in addition to
	sharpening programming skills.}

	\begin{itemize}
		\vspace{-0.5em}\item Built an automatic choral music generator. Utilizing artificial intelligence techniques and music theory led to an elegant and extendable framework that consistently outputs pleasing music.
		(Python)
        \vspace{-0.5em}\item Automatic handwriting recognition. Utilized adaptive spline models to model and recognize handwritten characters. (Matlab)
        \vspace{-0.5em}\item Maintains a personal website that discusses topics related to vision and learning.
        \vspace{-0.5em}\item Implemented an efficient barcode decoder
        for the Interleaved 2-of-5 format. (Python, OpenCV)
		\vspace{-0.5em}\item Built a Python 2.5 compiler targeting the x86 ISA, with the addition of
		optional strong typing support. (C++, Bison variant, Python)
		\vspace{-0.5em}\item Built a spam classifier, utilizing the Naive Bayes probabilistic model in order to make
		learning and inference stages tractable. (Python, numpy)
		\vspace{-0.5em}\item Implemented several AI algorithms for the game of Pacman, utilizing
		techniques such as: graph search, adversarial search, and reinforcement learning. (Python)
	\end{itemize}
	\end{position}
\end{component}

\vspace{-2.0em}

\begin{component}{Key Skills}
	\textbf{Technical Skills}\\
		Programming Languages: Python, Matlab, C, C++, Java, Javascript, HTML, CSS, php, Scheme, Assembly (MIPS, x86\_64)\\
        Specializations: Face recognition, computer vision, machine learning, nonlinear optimization, data analysis.\\
%		Environments: UNIX, Windows\\
		Other: OpenCV, vlfeat, numpy, scipy, Django, OpenGL \\
		Productivity: Version control (svn, git, mercurial), GDB, Wireshark, UNIX toolset, LaTeX
\end{component}

\begin{component}{Graduate Coursework}
\begin{tabularfw}{l c r}
Machine Perception & Linear Programming & Convex Optimization \\
Reasoning with Partial Beliefs & Large Scale Optimization & Applied Probability \\
&Visual Modeling \\
\end{tabularfw}
\end{component}

\begin{component}{Undergraduate Coursework}
	\begin{tabularfw}{l c r}
	Algorithms & Discrete Math & Data Structures \\
	Machine Structures & Operating Systems & Artificial Intelligence \\
	Machine Learning & Compilers & Computer Security \\
    Linear Algebra & Computer Networks & Computer Graphics \\
	\end{tabularfw}
\end{component}

\begin{comment}
\begin{component}{Relevant Coursework}
	\begin{tabularfw}{l c c r}
	Algorithms & Discrete Math & Data Structures & Machine Structures \\
	Operating Systems & Artificial Intelligence & Machine Learning \\
	Compilers & Computer Security
	\end{tabularfw}
\end{component}
\end{comment}

\end{document}

