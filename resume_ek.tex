\documentclass{resume}
\usepackage{hyperref}
\hypersetup{colorlinks,urlcolor=blue}
\usepackage{url}

\name{Eric Kim}
\addressone{248 Guerrero Street, Apt 202}
\addresstwo{San Francisco, CA 94103}
\email{eric.kim.cs@gmail.com}
\phone{805-300-9474}
\website{\href{http://www.eric-kim.net}{http://www.eric-kim.net}}

\usepackage{verbatim}
\geometry{margin=0.4in}

\begin{document}

\maketitle
\thispagestyle{empty} %% no page numbers

\vspace{-0.5em}

%% \begin{component}{Objective}
%% To develop innovative products with particular focus in utilizing visual data to enable exciting applications.
%% Interests: computer vision, face recognition, machine learning, analytics.\\
%% \end{component}

\vspace{-0.5em}

\begin{component}{Education}
    \school{University of California, Los Angeles}{Computer Science}{M.S.}{2013 -- 2016}

    \hspace{1em} Advisors: Professor Demetri Terzopoulos, Dr. M. Alex O. Vasilescu (Tensor Vision Technologies)

    \hspace{1em} Thesis: ``{A Part-Based, Multiresolution, TensorFaces Approach to Image-Based Facial Verification}''
    
	\school{University of California, Berkeley}{Computer Science}{B.A.}{2007 -- 2011}
\end{component}

\vspace{-0.25em}

\begin{component}{Key Skills}
		Languages: Python, C/C++, Scala, Matlab, Java, Javascript, HTML, CSS, PHP, Scheme, x86\_64, MIPS.\\
        Specializations: Computer vision, deep learning, face recognition, machine learning, medical imaging.\\
%		Environments: Unix, Windows. \\
		Libraries: Caffe, Caffe2, Tensorflow, pytorch, numpy, scipy, OpenCV, Spark \\
%		Productivity: Unix toolset, version control (svn, git, mercurial), gdb, LaTeX.
\end{component}

\vspace{-1.5em}

\begin{component}{Experience}
  \vspace{0.25em}
  \begin{position}{Software Engineer}{January 2017 -- Present}{Pinterest, Inc.}{Discovery, Visual Search}
    {As lead of the object detection group, I apply the state of the art in computer vision to gain a rich understanding of the visual contents of each Pin. Implemented and launched a product feature ``Lens your Look'' that unifies text search with visual search to recommend outfits, and wrote a \href{https://medium.com/@Pinterest_Engineering/building-lens-your-look-unifying-text-and-camera-search-1b2f3ef4e393}{blog post} describing the technical work.
      Designed and implemented a scalable, efficient Spark feature extraction pipeline that extracts visual signals on the billions of Pinterest images within hours. }
  \end{position}
  
	\begin{position}{Graduate Researcher, Intern}{September 2014 -- June 2016}{University of California, Los Angeles}{Department of Computer Science}
	\emph{Tensor Vision Technologies}

    {Analyzed faces in a multiresolution, part-based multilinear framework, and improved face verification results by 13\% on the ``Labeled Faces in the Wild'' dataset relative to previous multilinear work (79\% overall).
%This project became my MS thesis.
}
	\end{position}

\vspace{-0.25em}

    \begin{position}{Research Programmer}{January 2016 -- December 2016}{University of California, Los Angeles}{School of Dentistry}
{
Developed a statistical model of shape and appearance to perform bone contour segmentation of 3D medical imaging data.
%The algorithm iteratively improves the contour by updating each landmark based on a learned model of local appearance and global shape constraints.
%Enhanced accuracy of the model by extending the appearance model and the search algorithm to work well on 3D data.
\\
Quantitatively determined statistically significant facial surgery effects on facial structure.
%After achieving mesh correspondence via a nonrigid iterative closest point registration algorithm, I ran statistical tests on pre/post operation facial structure data to determine statistically significant regions of change.
This work led to a publication.
}
      \end{position}

\vspace{-0.75em}

	\begin{position}{Research Assistant}{May 2011 -- August 2013}
	 	{University of California, Berkeley}{Department of Computer Science}
	{Led the development of an open-source election auditing software: OpenCount.
     Utilized computer vision for automatic ballot tallying: image registration, digit recognition, barcode decoding.
     Successfully performed
         pilot audits in California counties.
}
	\end{position}

\vspace{-0.25em}

	\begin{position}{Teaching Assistant}{May 2010 -- June 2016}
		{University of California, Berkeley}{Department of Computer Science}
    \emph{University of California, Los Angeles}

	{Taught undergraduate computer science courses, spanning: Python, Scheme, Java, C, C++, and x86\_64.
	  Duties included holding sections, developing course materials, grading, and supervising office hours.
      \href{http://eric-kim.net/eric-kim-net/teaching.html}{Additional teaching details here}.
    }
	\end{position}
\end{component}

\vspace{-1.0em}

\begin{component}{Additional Projects}
	\begin{itemize}
		\vspace{-0.5em}\item \href{https://github.com/erickim555/FourVoices}{\emph{FourVoices}}: An automatic music generator.
%Utilizing artificial intelligence and music theory led to an elegant and extendable framework that consistently outputs pleasing music.
Using principles of music theory, I transformed the music generation problem into a set of constraints and variables, which I solve with a general-purpose constraint satisfaction solver.
Hosted on GitHub, the project features a wiki and tutorials on usage.
		(Python)
%        \vspace{-0.5em}\item Handwriting recognition. Utilized adaptive splines to recognize handwritten characters. (Matlab)
        \vspace{-0.5em}\item \href{http://eric-kim.net/cs269\_fa2014/index.html}{Handwriting recognition}:
        Implemented an adaptive deformable spline model to recognize handwritten characters using appearance and shape information.
        %To recognize a handwritten character, an adaptive deformable spline model is fit to the character via an iterative deformation algorithm that integrates both apperance and shape information.
%The algorithm outputs a deformation cost which is used for recognition: the label of the spline model with smallest cost is declared the output label.
(Matlab)
%        \vspace{-0.5em}\item Efficient barcode decoder
%	        for the Interleaved 2-of-5 format. (Python, OpenCV)
            \vspace{-0.5em}\item Wrote a popular tutorial on kernel methods as used in machine learning: \href{http://eric-kim.net/eric-kim-net/posts/1/kernel\_trick\_blog\_ekim\_12\_20\_2017.pdf}{``\emph{Everything You Wanted to Know about the Kernel Trick (But Were Too Afraid to Ask)}''}.
            %This article is the second Google search result for ``kernel trick'', as of 2016.
		%\vspace{-0.5em}\item Python 2.5 compiler targeting the x86 ISA, with the addition of
		%strong typing support. (C++, Python)
		%\vspace{-0.5em}\item Implemented various key low-level aspects of the 
		%educational NachOS operating system, such as demand paging, multi-threading, virtual memory, and 
		%networking.
		%\vspace{-0.5em}\item Built a spam classifier, utilizing the Naive Bayes probabilistic model. (Python, numpy)
		%\vspace{-0.5em}\item Implemented several AI algorithms for the game of Pacman, utilizing
		%techniques such as: graph search, adversarial search, and reinforcement learning. (Python)
	\end{itemize}
\end{component}

\vspace{-0.75em}

\begin{component}{Academic Papers}
\vspace{-0.25em}
\begin{itemize}
\item[] ``Three-dimensional soft tissue analysis of the face following micro-implant-supported maxillary skeletal expansion,'' Sara Abedini, Islam Elkenawy, Eric Kim, Won Moon. \emph{Progress in Orthodontics}, 2018. \emph{(Accepted, publication pending)}
  \vspace{-0.5em}
        \item[] ``Improved Support for Machine-Assisted Ballot-Level Audits,'' Eric Kim, Nicholas Carlini, Andrew Chang, George Yiu, Kai Wang, David Wagner. \emph{EVT/WOTE 2013}, August 2013.
\vspace{-0.5em}
        \item[] ``Operator-Assisted Tabulation of Optical Scan Ballots,'' Kai Wang, Eric Kim, Nicholas Carlini, Ivan Motyashov, Daniel Nguyen, David Wagner. \emph{EVT/WOTE 2012}, August 2012.
\vspace{-0.5em}
		\item[] ``An Analysis of Write-in Marks on Optical Scan Ballots,'' Theron Ji, Eric Kim, Raji Srikantan, Alan Tsai, Arel Cordero, and David Wagner. \emph{EVT/WOTE 2011}, August 2011.
\vspace{-0.5em}
	\end{itemize}
\end{component}

%% \begin{component}{Relevant Coursework}
%% 	\begin{tabularfw}{l c r}
%% 	Algorithms & Discrete Math & Data Structures \\
%% 	Machine Structures & Operating Systems & Artificial Intelligence \\
%% 	Machine Learning & Compilers & Computer Security \\
%% 	Convex Optimization & Machine Perception & Computer Networking
%% 	\end{tabularfw}
%% \end{component}

%% \begin{comment}
%% \begin{component}{Relevant Coursework}
%% 	\begin{tabularfw}{l c c r}
%% 	Algorithms & Discrete Math & Data Structures & Machine Structures \\
%% 	Operating Systems & Artificial Intelligence & Machine Learning \\
%% 	Compilers & Computer Security
%% 	\end{tabularfw}
%% \end{component}
%% \end{comment}

\end{document}

