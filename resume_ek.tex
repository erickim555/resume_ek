\documentclass{resume}
\name{Eric Kim}
\addressone{417 1/2 Veteran Avenue}
\addresstwo{Los Angeles, CA 90024}
\email{eric.kim.cs@gmail.com}
\phone{805-300-9474}
\website{http://www.eric-kim.net}

\usepackage{verbatim}
\geometry{margin=0.4in}

\begin{document}

\maketitle
\thispagestyle{empty} %% no page numbers

\vspace{-0.5em}

%% \begin{component}{Objective}
%% To develop innovative products with particular focus in utilizing visual data to enable exciting applications.
%% Interests: computer vision, face recognition, machine learning, analytics.\\
%% \end{component}

\vspace{-0.5em}

\begin{component}{Education}
    \school{University of California, Los Angeles}{Computer Science}{M.S.}{2013 -- 2016}

    \hspace{1em} Advisors: Professor Demetri Terzopoulos, Dr. M. Alex O. Vasilescu (Tensor Vision Technologies)

    \hspace{1em} Thesis: ``\emph{A Part-Based, Multiresolution, TensorFaces Approach to Image-Based Facial Verification}''
    
	\school{University of California, Berkeley}{Computer Science}{B.A.}{2007 -- 2011}
\end{component}

\vspace{-0.25em}

\begin{component}{Key Skills}
		Languages: Python, Matlab, Java, C, C++, Javascript, HTML, CSS, PHP, Scheme, x86\_64, MIPS.\\
        Specializations: Computer vision, face recognition, machine learning, medical imaging, nonlinear optimization.\\
%		Environments: Unix, Windows. \\
		Libraries: OpenCV, vlfeat, numpy, scipy, OpenGL \\
		Productivity: Unix toolset, version control (svn, git, mercurial), gdb, Wireshark, LaTeX.
\end{component}

\vspace{-0.25em}

\begin{component}{Experience}
	\begin{position}{Graduate Researcher/Intern}{September 2014 -- June 2016}{University of California, Los Angeles}{Department of Computer Science}
	\emph{Tensor Vision Technologies}

    {Analyzed faces in a multiresolution, part-based multilinear framework, and improved verification results by 13\% on the ``Labeled Faces in the Wild'' dataset relative to previous multilinear work (79\% overall).
This work matured into my MS thesis.
}
	\end{position}

\vspace{-0.25em}

    \begin{position}{Research Programmer}{January 2016 -- Present}{University of California, Los Angeles}{School of Dentistry}
{
Developed a statistical model of shape and appearance to perform bone contour segmentation of 3D imaging data.
The algorithm iteratively improves the contour by updating each landmark based on a learned model of local appearance, followed by a global shape constraint update.
\\
Applied 3D mesh algorithms to quantitatively determine facial surgery effects on facial structure.
After achieving mesh correspondence by applying a nonrigid iterative closest point registration algorithm, I ran statistical tests on pre/post operation facial structure data to determine statistically significant regions of change.
}
      \end{position}

\vspace{-0.25em}

	\begin{position}{Research Assistant}{May 2011 -- August 2013}
	 	{University of California, Berkeley}{Department of Computer Science}
	{Spearheaded the development of an open-source election auditing software: OpenCount.
     Utilized computer vision for automatic ballot tallying: image registration, digit recognition, and barcode decoding.
     Successfully performed
         several pilot audit programs in California counties.
}
	\end{position}

\vspace{-0.25em}

	\begin{position}{Teaching Assistant}{May 2010 -- June 2016}
		{University of California, Berkeley}{Department of Computer Science}
    \emph{University of California, Los Angeles}

	{Taught undergraduate computer science courses, spanning: Python, Scheme, Java, C, C++, and x86\_64.
	 Duties included holding sections, developing course materials, grading, and supervising office hours.}
	\end{position}
\end{component}

\vspace{-1em}

\begin{component}{Additional Projects}
	\begin{itemize}
		\vspace{-0.5em}\item \emph{FourVoices}: An automatic music generator.
%Utilizing artificial intelligence and music theory led to an elegant and extendable framework that consistently outputs pleasing music.
Using principles of music theory, I transformed the music generation problem into a set of constraints and variables, which I then solve with a general-purpose constraint satisfaction solver.
Hosted on GitHub, the project features a wiki and illustrated tutorials on usage.
		(Python)
%        \vspace{-0.5em}\item Handwriting recognition. Utilized adaptive splines to recognize handwritten characters. (Matlab)
        \vspace{-0.5em}\item Handwriting recognition. Handwriting is modeled as instantiations of deformable spline models.
To recognize a handwritten character, a deformable spline model is fit to the character via an iterative deformation algorithm.
%The algorithm outputs a deformation cost which is used for recognition: the label of the spline model with smallest cost is declared the output label.
(Matlab)
        \vspace{-0.5em}\item Efficient barcode decoder
	        for the Interleaved 2-of-5 format. (Python, OpenCV)
        \vspace{-0.5em}\item Wrote a gentle tutorial to kernel methods as used in machine learning. Title: ``\emph{Everything You Wanted to Know about the Kernel Trick (But Were Too Afraid to Ask)}''. This article is the second search result returned for Google searches of ``kernel trick'', as of 2016.
		\vspace{-0.5em}\item Python 2.5 compiler targeting the x86 ISA, with the addition of
		strong typing support. (C++, Python)
		%\vspace{-0.5em}\item Implemented various key low-level aspects of the 
		%educational NachOS operating system, such as demand paging, multi-threading, virtual memory, and 
		%networking.
		%\vspace{-0.5em}\item Built a spam classifier, utilizing the Naive Bayes probabilistic model. (Python, numpy)
		%\vspace{-0.5em}\item Implemented several AI algorithms for the game of Pacman, utilizing
		%techniques such as: graph search, adversarial search, and reinforcement learning. (Python)
	\end{itemize}
\end{component}

\vspace{-0.75em}

\begin{component}{Academic Papers}
\vspace{-0.5em}
	\begin{itemize}
        \item[] ``Improved Support for Machine-Assisted Ballot-Level Audits,'' Eric Kim, Nicholas Carlini, Andrew Chang, George Yiu, Kai Wang, David Wagner. \emph{EVT/WOTE 2013}, August 2013.
\vspace{-0.5em}
        \item[] ``Operator-Assisted Tabulation of Optical Scan Ballots,'' Kai Wang, Eric Kim, Nicholas Carlini, Ivan Motyashov, Daniel Nguyen, David Wagner. \emph{EVT/WOTE 2012}, August 2012.
\vspace{-0.5em}
		\item[] ``An Analysis of Write-in Marks on Optical Scan Ballots,'' Theron Ji, Eric Kim, Raji Srikantan, Alan Tsai, Arel Cordero, and David Wagner. \emph{EVT/WOTE 2011}, August 2011.
\vspace{-0.5em}
	\end{itemize}
\end{component}

%% \begin{component}{Relevant Coursework}
%% 	\begin{tabularfw}{l c r}
%% 	Algorithms & Discrete Math & Data Structures \\
%% 	Machine Structures & Operating Systems & Artificial Intelligence \\
%% 	Machine Learning & Compilers & Computer Security \\
%% 	Convex Optimization & Machine Perception & Computer Networking
%% 	\end{tabularfw}
%% \end{component}

%% \begin{comment}
%% \begin{component}{Relevant Coursework}
%% 	\begin{tabularfw}{l c c r}
%% 	Algorithms & Discrete Math & Data Structures & Machine Structures \\
%% 	Operating Systems & Artificial Intelligence & Machine Learning \\
%% 	Compilers & Computer Security
%% 	\end{tabularfw}
%% \end{component}
%% \end{comment}

\end{document}

