\documentclass{resume}
\name{Eric Kim}
\addressone{417 1/2 Veteran Avenue}
\addresstwo{Los Angeles, CA 90024}
\email{eric.kim.cs@gmail.com}
\phone{805-300-9474}
\website{http://www.eric-kim.net}

\usepackage{verbatim}

\begin{document}

\maketitle
\thispagestyle{empty} %% no page numbers

\vspace{-0.5em}

\begin{component}{Objective}
To develop innovative products with particular focus in utilizing visual data to enable exciting applications.
Interests: computer vision, face recognition, machine learning, analytics.\\
\end{component}

\vspace{-0.5em}

\begin{component}{Education}
    \school{University of California, Los Angeles}{Computer Science}{M.S.}{2016}

    \hspace{2em} Advisors: Professor Demetri Terzopoulos, Dr. M. Alex O. Vasilescu (Tensor Vision Technologies)
    
	\school{University of California, Berkeley}{Computer Science}{B.A.}{2011}
\end{component}

\vspace{0.5em}

\begin{component}{Key Skills}
		Languages: Python, Matlab, Java, C, C++, Javascript, HTML, CSS, PHP, Scheme, x86\_64.\\
        Specializations: Computer vision, face recognition, machine learning, nonlinear optimization.\\
%		Environments: Unix, Windows. \\
		Productivity: Unix toolset, version control (svn, git, mercurial), Django, gdb, Wireshark, LaTeX.
\end{component}

\vspace{0.5em}

\begin{component}{Experience}
	\begin{position}{Graduate Researcher/Intern}{September 2014 -- June 2016}{University of California, Los Angeles}{Department of Computer Science}
	\emph{Tensor Vision Technologies}

		Developed a novel facial verification system within a multilinear framework, and validated on several public datasets.
Utilized techniques from computer vision, machine learning, and multilinear analysis.
	\end{position}

    \begin{position}{Programmer}{January 2016 -- Present}{University of California, Los Angeles}{School of Dentistry}
      Developed a system to perform automatic fiducial marker acquisition on 3D cone-beam computed tomography scans of human subjects for the purpose of 3D cephalometric analysis.
      Extending the active shape model to perform automatic landmark acquisition led to a flexible solution.
      \end{position}

	\begin{position}{Research Assistant}{May 2011 -- August 2013}
	 	{University of California, Berkeley}{Department of Computer Science}
	{Spearheaded the development of an open-source election auditing software,
     utilizing image processing and computer vision techniques. 
     Successfully performed
         several pilot audit programs in California counties.
     Developed web applications to ease the image annotation process of
     computer vision datsets.}
	\end{position}

	\begin{position}{Teaching Assistant}{May 2010 -- June 2016}
		{University of California, Berkeley}{Department of Computer Science}
    \emph{University of California, Los Angeles}

	{Taught undergraduate computer science courses, spanning: Python, Scheme, Java, C, C++, and x86\_64.
	 Duties included holding sections, developing course materials, grading, and supervising office hours.}
	\end{position}
\end{component}

\vspace{-0.5em}

\begin{component}{Projects}
	\begin{itemize}
		\vspace{-0.5em}\item Automatic music generator. Utilizing artificial intelligence techniques and music theory led to an elegant and extendable framework that consistently outputs pleasing music.
		(Python)
        \vspace{-0.5em}\item Efficient barcode decoder
	        for the Interleaved 2-of-5 format. (Python, OpenCV)
        \vspace{-0.5em}\item Maintains a personal website. Topics include: 
        computer science, machine learning, computer vision.
		\vspace{-0.5em}\item Python 2.5 compiler targeting the x86 ISA, with the addition of
		strong typing support. (C++, Python)
		%\vspace{-0.5em}\item Implemented various key low-level aspects of the 
		%educational NachOS operating system, such as demand paging, multi-threading, virtual memory, and 
		%networking.
		%\vspace{-0.5em}\item Built a spam classifier, utilizing the Naive Bayes probabilistic model. (Python, numpy)
		%\vspace{-0.5em}\item Implemented several AI algorithms for the game of Pacman, utilizing
		%techniques such as: graph search, adversarial search, and reinforcement learning. (Python)
	\end{itemize}
\end{component}

\vspace{-0.5em}

\begin{component}{Academic Papers}
\vspace{-0.5em}
	\begin{itemize}
        \item[] ``Improved Support for Machine-Assisted Ballot-Level Audits,'' Eric Kim, Nicholas Carlini, Andrew Chang, George Yiu, Kai Wang, David Wagner. \emph{EVT/WOTE 2013}, August 2013.
\vspace{-0.5em}
        \item[] ``Operator-Assisted Tabulation of Optical Scan Ballots,'' Kai Wang, Eric Kim, Nicholas Carlini, Ivan Motyashov, Daniel Nguyen, David Wagner. \emph{EVT/WOTE 2012}, August 2012.
\vspace{-0.5em}
		\item[] ``An Analysis of Write-in Marks on Optical Scan Ballots,'' Theron Ji, Eric Kim, Raji Srikantan, Alan Tsai, Arel Cordero, and David Wagner. \emph{EVT/WOTE 2011}, August 2011.
\vspace{-0.5em}
	\end{itemize}
\end{component}

%% \begin{component}{Relevant Coursework}
%% 	\begin{tabularfw}{l c r}
%% 	Algorithms & Discrete Math & Data Structures \\
%% 	Machine Structures & Operating Systems & Artificial Intelligence \\
%% 	Machine Learning & Compilers & Computer Security \\
%% 	Convex Optimization & Machine Perception & Computer Networking
%% 	\end{tabularfw}
%% \end{component}

%% \begin{comment}
%% \begin{component}{Relevant Coursework}
%% 	\begin{tabularfw}{l c c r}
%% 	Algorithms & Discrete Math & Data Structures & Machine Structures \\
%% 	Operating Systems & Artificial Intelligence & Machine Learning \\
%% 	Compilers & Computer Security
%% 	\end{tabularfw}
%% \end{component}
%% \end{comment}

\end{document}

